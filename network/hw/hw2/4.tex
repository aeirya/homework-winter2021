\problem{}
.

\subproblem{}
غلط.
به این شیوه عمل می‌کند که ابتدا درخواست می‌کند و فایل 
\text{html}
صفحه را دریافت می‌کند. و با دیدن رفرنس به عکس‌های ناموجود
برای دریافت آن عکس‌ها نیز جداگانه (برای هرکدام) یک درخواست دیگر می‌دهد
.

\subproblem{} 
بله می‌توان. از آن‌جایی که ارتباط ماندگار است می‌توان در یک ارتباط درخواست‌های متوالی داد
.

\subproblem{}
می‌تواند. اتفاقا می‌توان درخواست چنین پیامی هم داد. به این صورت که با درخواست 
\text{head}
تنها 
\text{header}
مربوط به فایل مورد نظر را دریافت می‌کنیم
.

\subproblem{}
غلط.
بلکه مشخص می‌کند سرور پاسخ را در چه زمانی ساخته و ارسال کرده است.