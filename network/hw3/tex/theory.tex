1-

\begin{latin}
\subsection*{sender}
\paragraph{data from above}
	\begin{itemize}
		\item if next available sequence number in window, send packet.
	\end{itemize}

\paragraph{timeout(n)}
	\begin{itemize}
		\item mark packet $n$ as received.
		\item if $n$ smallest unreceived packet, advance window base to next unreceived sequence number.
	\end{itemize}

\paragraph{NACK(n)}
	\begin{itemize}
		\item resend packet $n$, restart timer.
	\end{itemize}	
	
\hline

\subsection*{receiver}
\paragraph{packet $n$ in [text{rcvbase, rcvbase}+N-1] }
\begin{itemize}
	\item out-of-oder: buffer, NACK(n-1), set timer.
	\item in-order: deliver
	(deliver buffered, in order packets and advance window to not-yet-received packet).
	
\end{itemize}

\paragraph{otherwise}
	\begin{itemize}
		\item ignore
	\end{itemize}

\paragraph{timeout(n)}
	\begin{itemize}
		\item NACK(n)
		\item reset timer
	\end{itemize}
	
	
\end{latin}

هربار که درخواستی گم می‌شود باید گیرنده متوجه آن شود (در حین دریافت بسته های بعدی) و آن را دوباره بخواهد و اگر بسته بعدی بسیار بعد از بسته گم شده ارسال شده باشد زمان تکمیل بسته های ما هم بسیار طولانی می‌شود.
اما در شرایطی که درخواست‌های زیادی گم نمی‌شوند این پروتکل بسیار خوب عمل میکند و در اکثر مواقع نیازی به پاسخ ندارد.

2-
\begin{enumerate}[label = \alph*]
	\item مثلا درخواست دو فایل 
	$A$ و $B$
		را می‌دهیم 
و پاسخ 
$A$
یا گم می‌شود یا زمانی که باید به ما می‌رسید سیستم ما خاموش شده است.
زمانی که روشن می‌شویم همچنان منتظر پاسخ $A$ هستیم اما پاسخ $B$ را دریافت می‌کنیم.

یا مثلا وقتی که درخواست این دو فایل را می‌دهیم اما پاسخ $B$ زودتر می‌رسد.

	\item راه حل این است که به هر درخواست قبل ارسال یک شماره‌ای نسبت بدهیم 
	و سرور در پاسخ‌هایش مشخص کند که این پاسخ به چه فایلی است.
	و بدین ترتیب کلاینت می‌تواند تشخیص دهد پاسخ کدام فایل را گرفته است.
\end{enumerate}

3- 
\begin{enumerate}[label = \alph*]
	\item بسته سوم (۲۰۰) با موفقیت ارسال شده اما قبل از رسیدن پاسخ مثبت آن سرور تایم‌اوت شده و آن را دوباره می‌فرستد. و پاسخی مثبت اولیه کلاینت در زمانی بعد ۳۰۰ میرسد پس سرور دیگر برای بار سوم ارسال نمی‌کند. \\
	برای بسته چهارم نیز سناریو کاملا مشابه رخ می‌دهد و ده میلی ثانیه قبل رسیدن ack آن دوباره retransmit 
	می‌شود.	
\end{enumerate}

ج)

\begin{latin}
\begin{tabular}{ |p{1.2cm}|p{1.2cm}|p{1.2cm}|p{5.2cm}|p{1.5cm}|p{2cm}|  }
 \hline
Send &Receive
 &
 RTT & Estimated RTT & dif & DevRTT\\
 \hline
 0	&	200 & 200 & 100*7/8 + 200/8 = 112.5 & 87.5 & 21.875 \\
 10	&	220 & 210 & 112.5*7/8 + 210/8 = 124.7 & 85.3&  37.73 \\
 20	&	310  & - & - & - & -\\
 30	&	320 & - & - & - & -\\
 320 &	340 & 20 & 124.7 *7/8 + 20/8= 111.6 & 91.6 &  51.19 \\
 \hline
\end{tabular}
\end{latin}

قابل توجه است برای بسته های ۳ و ۴ که retransmit شدند دیگر EstimatedRTT محاسبه نکردیم.


د)

ه)

\begin{latin}
\begin{tabular}{ |p{3cm}||p{3cm}|p{3cm}|p{3cm}|  }
 \hline
 \multicolumn{1}{|c|}{RTT} &
 \multicolumn{3}{||c|}{Estimated RTT} \\
 \hline
 - & \alpha = 0 & \alpha = 0.125 & \alpha = 1\\
 \hline
 200 & 100	& 112.5 & 200 \\
 210 & 100 & 124.7 & 210 \\
 20 & 100 & 111.6 & 20 \\
  \hline
\end{tabular}
\end{latin}

از تغییرات زیاد 
\lr{estimated rtt}
جلوگیری می‌کند و مقداری بینابین حدس قبلی و  اخرین rtt میدهد.

\newpage
