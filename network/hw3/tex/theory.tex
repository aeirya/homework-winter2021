1-

\begin{latin}
\subsection*{sender}
\paragraph{data from above}
	\begin{itemize}
		\item if next available sequence number in window, send packet.
	\end{itemize}

\paragraph{timeout(n)}
	\begin{itemize}
		\item mark packet $n$ as received.
		\item if $n$ smallest unreceived packet, advance window base to next unreceived sequence number.
	\end{itemize}

\paragraph{NACK(n)}
	\begin{itemize}
		\item resend packet $n$, restart timer.
	\end{itemize}	
	
\hline

\subsection*{receiver}
\paragraph{packet $n$ in [text{rcvbase, rcvbase}+N-1] }
\begin{itemize}
	\item test
\end{itemize}

%\paragraph{}

\paragraph{otherwise}
	\begin{itemize}
		\item ignore
	\end{itemize}

\paragraph{timeout(n)}
	hmm
	
\end{latin}

2-
\begin{enumerate}[label = \alph*]
	\item مثلا درخواست دو فایل 
	$A$ و $B$
		را می‌دهیم 
و پاسخ 
$A$
یا گم می‌شود یا زمانی که باید به ما می‌رسید سیستم ما خاموش شده است.
زمانی که روشن می‌شویم همچنان منتظر پاسخ $A$ هستیم اما پاسخ $B$ را دریافت می‌کنیم.

یا مثلا وقتی که درخواست این دو فایل را می‌دهیم اما پاسخ $B$ زودتر می‌رسد.

	\item راه حل این است که به هر درخواست قبل ارسال یک شماره‌ای نسبت بدهیم 
	و سرور در پاسخ‌هایش مشخص کند که این پاسخ به چه فایلی است.
	و بدین ترتیب کلاینت می‌تواند تشخیص دهد پاسخ کدام فایل را گرفته است.
\end{enumerate}

3- 
\begin{enumerate}[label = \alph*]
	\item بسته سوم (۲۰۰) با موفقیت ارسال شده اما قبل از رسیدن پاسخ مثبت آن سرور تایم‌اوت شده و آن را دوباره می‌فرستد. و پاسخی مثبت اولیه کلاینت در زمانی بعد ۳۰۰ میرسد پس سرور دیگر برای بار سوم ارسال نمی‌کند. \\
	بسته چهارم نیز 
	
\end{enumerate}}

\newpage
