
\newpage
\section{عنوان بخش}
شروع بحث جلسه فعلی. 

پاراگراف‌ها با یک خط خالی از هم جدا می‌شوند. لازم نیست بین هر دو پاراگراف از \verb|\par|  استفاده کنیم.

می‌توانید برای نوشتن فرمول چندخطی می‌توانید از محیط \verb|align| استفاده کنید:
\begin{align*} 
\min \quad \| f \|_{\infty}\\ 
f \in F_{s,t} 
\end{align*}

سعی کنید قواعد نگارش فارسی را رعایت کنید. از نیم‌فاصله
به درستی استفاده کنید. علامت نقل قول در فارسی بدین صورت است «». پس از نقطه و ویرگول و دونقطه و پرانتزبسته‌ای که قبل از نقطه نیست و این‌گونه علامت‌ها، یک فاصله بگذارید.

خوب است معادل انگلیسی اصطلاحات را در پاورقی%
\LTRfootnote{footnote}
بیاورید.

برای نوشتن انگلیسی در میان متن فارسی، از دستور \verb+\lr{}+ استفاده کنید. 
مثلا:
\lr{Some English text here} 
در میان متن درست می‌آید.

استفاده از تاکید به صورت 
\textbf{پررنگ}
کردن یا 
\textit{ایتالیک}
کردن مفید است. 

برای شبه کدها از پکیج
\lr{clrscode3e}
استفاده کنید. برای آشنایی با این پکیج
\lr{clrscode.pdf}
را مطالعه کنید.


\section{محیط‌های مختلف}
\begin{lemma}
	یک لم.
\end{lemma}
\begin{theorem}
	\label{thm:sample}
	یک قضیه. 
\end{theorem}
\begin{proof}
	بدیهی.
\end{proof}

\begin{example}
	یک مثال. 
\end{example}

ارجاع به قضیه 
\ref{thm:sample}.

ارجاع به مراجع
\cite{lecture1}
و
\cite{CLRS}.

%برای قاطی نشدن متن فارسی و انگلیسی از
% Enter
%  زیاد استفاده کنید!

\bibliographystyle{alpha}

\begin{thebibliography}{99}
	\bibitem{lecture1}
	جزوه جلسه اوّل
	\bibitem{dr. ghodsi}
	قدسی، محمّد. \textit{داده ساختارها و مبانی الگوریتم‌ها}. تهران: فاطمی، ۱۳۹۵
	
	\begin{latin}	%english reference
		
		%book in MLA style: https://owl.purdue.edu/owl/research_and_citation/mla_style/mla_formatting_and_style_guide/mla_works_cited_page_books.html
		\bibitem{CLRS}
		Cormen, Thomas H., et al.
		\textit{Introduction to Algorithms}. 
		3rd ed., MIT Press, 2009, pp. 18-22. %page numbers
	\end{latin}	
\end{thebibliography}
