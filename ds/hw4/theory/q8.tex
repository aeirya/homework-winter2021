الف)
فرض کنیم تعداد اعضایی که نگهداری می‌کنیم 
n
باشد.

دو حالت مرزی را بررسی می‌کنیم:
این که 
n 
طوری باشد که 
تمام آرایه های 
به طول 
توان ۰ تا 
k
از ۲ 
پر باشند.

و حالت دیگر این که 
تنها یک آرایه 
به طول 
$n = 2^k$
پر باشد.

در حالت اول مرتبه زمانی برای می‌شود با 
$T(n) = \sum_{i=1}^{k} log {2^i} = \sum_{i=1}^{k} i = O(k^2)$

و چون 
$\sum_{i=1}^{k} 2^i = 2^{k+1} - 1 = n$
داریم 
$k \in log(n)$ 
پس 
$T(n) = log^2 n$

در حالت دوم نیز 
$T(n) = log 2^k = log (n)$

هر عدد دیگر هم 
یک نتیجه بینابین این دو 
T 
دارد 
(بعضی از آرایه ها از اندازه های کوچک پر است و بعضی از آرایه های بزرگ)
و
در بدترین حالت 
مرتبه زمانی از 
$log^2 n$ 
می‌باشد.

ب)
\paragraph{بدترین حالت}
بدترین حالت زمانی رخ می‌دهد که 
$n= 2^{k}-1$
.
در این حالت تمام آرایه ها پر است 
و با اضافه شدن عضو 
جدید 
(آرایه به طول یک)
به طور زنجیری آرایه ها با هم 
ادغام می‌شوند 
تا آرایه به طول 
$2^k$
تشکیل شود.
هزینه ادغام هر دو آرایه با اندازه یکسان از مرتبه 
اندازه‌شان است و
اندازه هر آرایه هم توانی از دو است
پس خواهیم داشت:

\begin{equation*}
    T(n) = \sum_{i=0}^{k-1}2^i = 2^k-1 = n \in O(n)
\end{equation*}
\paragraph{تحلیل سرشکن}
عدد 
$M=2^k$
را نزدیک ترین عدد به
$N$
در نظر بگیریم که
$N < M$
.

می‌دانیم با اضافه کردن آخرین 
(
M
 امین
)
عضو 
در اخرین مرحله ادغام‌های زنجیری
آرایه ها 
باید دو
آرایه به اندازه 
$2^{k-1}$ 
با هم در مرتبه اندازه شان
( 
$2^k$
)
ادغام شوند. از طرفی می‌دانیم برای شکل گرفتن هر کدام از این
دو آرایه 
به اندازه
$2^{k-1}$ 
باید 
دو آرایه با اندازه 
$2^{k-2}$ 
در مرتبه 
$2^{k-1}$ 
با هم ادغام شوند 
.
پس برای دو آرایه ادغام شونده به 
$2^{k}$ 
جمعا 
$2*2^{k-1} = 2^k$
هزینه می‌شود. 
برای ساختن خود آرایه های اندازه کوچک تر نیز به طریق مشابه خواهد بود که مرتبه هزینه ادغام نصف اما تعداد آرایه ها به آن اندازه دوبرابر می‌شود و در نتیجه  
هزینه کل برابر خواهد بود با

\begin{equation*}
    T(N) \leq T(M) = \sum_{i=0}^{k} 2^k = k \cdot 2^k \xrightarrow{k = log M} M log M
\end{equation*}
هزینه سرشکن نیز برای اضافه کردن هر یک از 
$M$
عضو برابر با 
$\frac{MlogM}{M}= logM$
خواهد بود
.
اضافه کردن 
N 
عضو 
از مرتبه مشابهی خواهد بود
.

