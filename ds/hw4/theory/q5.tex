رشته‌ها را به شکل مرتب شده بر اساس طول و با نماد 
$w_1, \cdots, w_n$
نشان می‌دهیم که 
$n$ 
تعداد رشته ها است و 
طول هرکدام به شکل 
$l_i$
و داریم  
$l_i < l_{i+1}$
.

فرض کنیم 
$m_1$
پرسمان از رشته‌های جفت طول کوچک‌تر مساوی  
$x$
یا 
یکی بزرگ‌تر و یکی کوچک‌تر از 
$x$
انجام دهیم و 
$m_2 = m - m_1$
پرسمان با هر دو رشته با طول بیشتر
.

مرتبه زمانی 
اجرای الگوریتم بر روی دو رشته از مرتبه طول رشته کوتاه‌تر است پس برای 
$m_1$
پرسش اول داریم 
$T = O(m_1 \cdot x)$
.

برای 
$m_2$
پرسش نوع دوم به شکل زیر استدلال می‌کنیم
:

از آن‌جایی که طول تمام رشته ها از 
x
بزرگ‌تر است و
جمع طول آن‌ها حداکثر 
k 
می‌شود پس حداکثر 
تعداد آن‌ها
$n_2 = \frac{k}{x}$
می‌باشد
.

از طرفی
اگر 
$n_2 = \frac{k}{x}$
و 
$n_1 = n - n_2$
را تعداد رشته‌های با طول بزرگ‌تر و کوچک‌تر از 
$x$
در نظر بگیریم٫
در صورت وجود محدودیت برای پرسمان تکراری حداکثر
$n_2\choose 2$
زوج
رشته از نوع دو می‌توان انتخاب کرد
که از 
$O((n_2)^2)$
می‌باشد
و
مشابها تعداد زوج‌های قابل انتخاب از رشته‌های نوع اول 
از مرتبه 
$O((n_1)^2)$
.

پس مرتبه زمانی برای پرسش‌های نوع دوم به شکل 
$O(m_2 \cdot k) = O(n_2^2\cdot k) = O((\frac{k}{x})^2 \cdot k) = O(\frac{k^3}{x^2})$

پس مجموع هزینه‌ای که برای این دو نوع پرسش خواهیم داد به شکل زیر خواهد بود
:
\begin{equation*}
    T = O(x (n-n_1)^2) + O(k(n_2)^2) = O(x (n-\frac{k}{x})^2) + O(\frac{k^3}{x^2})
\end{equation*}
