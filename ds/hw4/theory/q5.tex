رشته‌ها را به شکل مرتب شده بر اساس طول و با نماد 
$w_1, \cdots, w_n$
نشان می‌دهیم که 
$n$ 
تعداد رشته ها است و 
طول هرکدام 
$l_i$
و داریم  
$l_i < l_{i+1}$
.

فرض کنیم 
$m_1$
پرسمان از رشته‌های جفت طول کوچک‌تر مساوی  
$x$
یا 
یکی بزرگ‌تر و یکی کوچک‌تر از 
$x$
انجام دهیم و 
$m_2 = m - m_1$
پرسمان با هر دو رشته با طول بیشتر
.

مرتبه زمانی 
اجرای الگوریتم بر روی دو رشته از مرتبه طول رشته کوتاه‌تر است پس برای 
$m_1$
پرسش اول داریم 
$T = O(m_1 \cdot x)$
.

برای 
$m_2$
پرسش نوع دوم به شکل زیر استدلال می‌کنیم
:

تنها 
یک زوج
$(w_n,w_n)$
داریم که طول رشته کوچک‌ترش برابر با 
$l_n$
است و دو زوج
که 
طول رشته کوچک‌ترش برابر با 
$l_{n-1}$
و \dots.
هزینه پرسمان‌های نوع دو به شکل 
$T = l_{n} + 2 l_{n-1} + \cdots + (y+2)l_{n-y+1}$
می‌شود
که 
$\sum_{i=1}^{y+2} 1 = m_2$
یعنی 
$y \in \Theta(\sqrt{m_2})$
.


رابطه بالا در صورتی بیشینه می‌شود که 
$l_n = l_{n-1} = \cdots = l_{n-y+1}$

و حداکثر مقدار 
$l_i \left(i \in \interval[n-y+1, n] \right)$
برابر می‌شود با وقتی که 
$n_1 = 0$
و
$n_2 - y$
رشته به طول 
$x$ 
و 
$y$ 
رشته به طول 
\newline
$l_i = \frac{k-x(n_2-y)}{y}$
داشته باشیم 
.
پس 
$l_i \in O(\frac{k}{y}+x)$
و داریم 
$y \in \Theta (\sqrt m_2)$
پس 
$l_i$
به شکل 
$l_i \in O(\frac{k}{\sqrt{m_2}}+x)$
است
.   

و
برای مرتبه زمانی 
$m_2$
پرسش‌ نوع دو خواهیم داشت
$T = l_i (1 + 2 + \cdots + (y+2)) \in l_i \cdot O(y^2) = O((\frac{k}{\sqrt m_2}+x)m_2)$

از آن‌جایی که طول تمام رشته ها (در نوع دو)‌ از 
x
بزرگ‌تر است و
جمع طول آن‌ها حداکثر 
k 
می‌شود پس حداکثر 
تعداد آن‌ها
$n_2 = \frac{k}{x}$
می‌باشد
.

از طرفی
اگر 
$n_2 = \frac{k}{x}$
و 
$n_1 = n - n_2$
را تعداد رشته‌های با طول بزرگ‌تر و کوچک‌تر از 
$x$
در نظر بگیریم٫
در صورت وجود محدودیت برای پرسمان تکراری حداکثر
$n_2\choose 2$
زوج
رشته از نوع دو می‌توان انتخاب کرد
که از 
$O((n_2)^2)$
می‌باشد
و
مشابها تعداد زوج‌های قابل انتخاب از رشته‌های نوع اول 
از مرتبه 
$O((n_1)^2)$
.

% پس مرتبه زمانی برای پرسش‌های نوع دوم به شکل 
% $O(m_2 \cdot k) = O(n_2^2\cdot k) = O((\frac{k}{x})^2 \cdot k) = O(\frac{k^3}{x^2})$

پس مجموع هزینه‌ای که برای این دو نوع پرسش خواهیم داد به شکل زیر خواهد بود
:
\begin{equation*}
    T = O(x (n-n_1)^2) + O(k(n_2)^2) = O(x (n-\frac{k}{x})^2) + O((\frac{k}{\sqrt m_2}+x)m_2)
\end{equation*}
% اگر از رابطه بر اساس 
% x 
% مشتق بگیریم 
% خواهیمم داشت:
% \begin{equation*}
%     \frac{dT}{dx} = O(n^2) + 
% \end{equation*}

به ازای 
$ x= \sqrt k$
خواهیم داشت
\begin{equation*}
    T \in O((n -\sqrt k)^2 \sqrt k + k\sqrt m_2 + m_2\sqrt k) = O((n^2+k+m_2) \sqrt k + k \sqrt m_2) = O((n^2+k+n_2^2) \sqrt k + k n_2)
\end{equation*}

ب)

کافیست که پاسخ ها را ذخیره کنیم.
درختی دودویی متوازن نگهداری میکنیم که هر گره آن نشان دهنده اندیس رشته اول یک پرسمان است و در آن درخت دیگری نگهداری می‌شود که هر گره
نشان هنده اندیس دوم پرسمان است
و
محتوای ان پاسخ این پرسمان است.

از آن‌جایی که 
m 
پرسمان تنها داریم و ارتفاع 
هر دو درخت 
حداکثر 
از 
O(m)
است 
(در هر پرسمان یک نتیجه به آن اضافه می‌شود)
و جستجو در درخت از مرتبه ارتفاع است
پس 
مرتبه زمانی 
$O(m log^2 m)$
زمان به زمان الگوریتم اصلی اضافه می‌شود
.
از طرفی 
$m \leq n \choose 2 \in O(n^2)$
پس 
زمان اضافه شده می‌شود
$O(m log^2 n^2) = O(mlog^2n) = O(mlogK)$