\par
الف)
هر یک از بخش‌های لیست اندازه 
$k$
دارند و در بدترین حالت در 
$O(k^2)$
مرتب می‌شوند.
جمعا 
$n/k$
لیست داریم پس روی هم 
$O(\frac{n}{k}\cot k^2) = O(nk)$
زمان می‌گیرند.

\par
%ب)
%دو لیست مرتب شده با سایز
%$k$
%با حداکثر 
%$2k-1$
%مقایسه ادغام می‌شوند (تمرین سری ۲).
%
%اگر $m$ لیست مرتب شده با اندازه $k$ داشته باشیم٫ ادغام آن ها
%$T(n = mk) = O(mk\cdot log(mk))$
%به طول می‌انجامد.
%پس اگر 
%$m = \frac{n}{k}$
%داریم:
%\begin{equation*}
%	T(n) = O(\frac{n}{k}\cdot k log(\frac{n}{k}\cdot k)) = O(
%\end{equation*}
ب)
تقسیم یک لیست به $n/k$ لیست مرتب مانند این است که بخشی از عمق الگوریتم مرج سورت را طی کرده باشیم (تا وقتی که اندازه هر لیست k شده باشد). در هر مرحله اندازه زیرلیست‌ها دوبرابر می‌شود پس انگار $log.k$ مرحله جلوییم.
و در نتیجه تعداد عملیات باقی مانده متناسب با عمق باقی مانده است که می‌شود 
$log \cdot n - log \cdot k = log \frac{n}{k}$
.
در هر مرحله نیز از 
$\Theta(n)$ 
کار انجام می‌دهیم پس مرتبه زمانی این ادغام از 
$\Theta(nlog\frac{n}{k})$
خواهد بود.

ج)
k 
نمی‌تواند از 
$\Omega(n)$ 
باشد چرا که 
الگوریتم ما از مرتبه 
$\Omega(n^2)$
خواهد شد.
به ازای 
$k= O(log(n))$
خواهیم داشت
$$T = O(n\cdot logn + n log \frac{n}{logn}) = O(nlogn - n log(log(n))) = O(nlogn)$$

پس $k(n) \in O(logn)$
.

د)
اگر
$a,b$
ثابت باشند
داریم 
$T(n,k) = ank + bnlog\frac{n}{k} = ank + bnlogn - bnlogk$.
قرار می‌دهیم 
$\frac{d}{dk} T = 0$
که می‌دهد 
$$ an + 0 - bn/k = 0 \rightarrow k = b/a $$

یعنی یک مقدار بهینه ثابت برای k وجود داره.
 می‌توانیم برای $n$ های نسبتا بزرگ $k$ های مختلف 
 از
 $O(logn)$
 را امتحان کنیم و بهترین مورد را انتخاب کنیم.