% این تمپلیت از درس ساختمان داده ی دکتر علیمی-پاییز۹۹ برداشته شده است.

\documentclass[11pt]{article}
\usepackage{pgfplots}

\usepackage{tikz}
\usetikzlibrary{datavisualization}
\usetikzlibrary{datavisualization.formats.functions}


%You should edit HW.tex, not this file!
\usepackage{mathtools, amsmath, nccmath}
\usepackage{bm}
\usepackage{amsthm}
\usepackage{latexsym}
\usepackage{amssymb}
\usepackage{verbatim}
\usepackage{enumitem,array}
\usepackage{tikz}
\usepackage{color}
\usepackage{bookmark}
\usepackage{geometry}
\geometry{
 a4paper,
 right=15mm,
 left=15mm,
 top = 10mm,
 bottom = 20mm
}
\usepackage{listings}
\usepackage{bussproofs} %for prooftrees
\usepackage{hyperref}
\hypersetup{
	colorlinks=true,
	linkcolor=blue,
	filecolor=magenta,      
	urlcolor=cyan,
}
\usepackage{xepersian}

\definecolor{light-gray}{gray}{0.98}
\definecolor{blue-green}{rgb}{0,0.6,0.5}
\lstset{ 
  backgroundcolor=\color{light-gray},   % choose the background color; you must add \usepackage{color} or \usepackage{xcolor}; should come as last argument
  basicstyle=\footnotesize\color{violet},        % the size of the fonts that are used for the code
  breakatwhitespace=false,         % sets if automatic breaks should only happen at whitespace
  breaklines=true,                 % sets automatic line breaking
  %frame=lines
  captionpos=b,                    % sets the caption-position to bottom
  commentstyle=\itshape\color{blue-green},    % comment style
  %escapeinside={\%*}{*)},          % if you want to add LaTeX within your code
  extendedchars=true,              % lets you use non-ASCII characters; for 8-bits encodings only, does not work with UTF-8
  frame=single,	                   % adds a frame around the code
  keepspaces=true,                 % keeps spaces in text, useful for keeping indentation of code (possibly needs columns=flexible)
  keywordstyle=\bfseries\color{blue},       % keyword style
  %language=Octave,                 % the language of the code
  morekeywords={*,...},            % if you want to add more keywords to the set
  numbers=left,                    % where to put the line-numbers; possible values are (none, left, right)
  numbersep=5pt,                   % how far the line-numbers are from the code
  numberstyle=\tiny\color{gray}, % the style that is used for the line-numbers
  rulecolor=\color{light-gray},         % if not set, the frame-color may be changed on line-breaks within not-black text (e.g. comments (green here))
  showspaces=false,                % show spaces everywhere adding particular underscores; it overrides 'showstringspaces'
  showstringspaces=false,          % underline spaces within strings only
  showtabs=false,                  % show tabs within strings adding particular underscores
  stepnumber=1,                    % the step between two line-numbers. If it's 1, each line will be numbered
  stringstyle=\color{cyan},     % string literal style
  tabsize=2,	                   % sets default tabsize to 2 spaces
  %title=\lstname                   % show the filename of files included with \lstinputlisting; also try caption instead of title
}
%\setlist[enumerate,1]{start=0} %for 0-based enumeration

%\settextfont[
 %BoldFont={HM_XNiloofarBd.ttf}, 
 %ItalicFont={HM_XNiloofarIt.ttf},
 %BoldItalicFont={HM_XNiloofarBdIt.ttf}
 %]{HM_XNiloofar.ttf}
\settextfont{HM XNiloofar}
\ExplSyntaxOn \cs_set_eq:NN \etex_iffontchar:D \tex_iffontchar:D \ExplSyntaxOff
\setdigitfont{HM XNiloofar}
%\setdigitfont{ParsiDigits}
\defpersianfont\outline[Scale=1]{HM XNiloofar Outline}
\setlength{\parindent}{1.5em}
\setlength{\parskip}{0.9em}
\renewcommand{\baselinestretch}{1.5}

%for persian enumeration
\makeatletter
\def\@myharfi#1{\ifcase#1\or آ\or ب\or پ\or ت\or ث\or
ج\or چ\or ح\or خ\or د\or ذ\or ر\or ز\or س\or ش\or ص\or ض\or ع\or غ\or
ف\or ق\or ک\or گ\or ل\or م\or ن\or و\or ه\or ی\else\@ctrerr\fi}
\def\myharfi#1{\expandafter\@myharfi\csname c@#1\endcsname}
\makeatother
\AddEnumerateCounter{\myharfi}{\@myharfi}{}

\newcommand{\lecture}[4]{
%\pagestyle{empty}

	%\begin{center}
	%		\vspace{-1cm}
	%	   \includegraphics[scale=0.15]{Sharif}%\hfill \\[1em]  
	%\end{center}
	%\vspace{-3em}
\begin{center}

\bf
%\begin{outline} 
{
\LARGE
#1
}
%\end{outline} 
\\
تمرین #2
\end{center}
\vspace*{-1em}
\noindent
نام و نام‌خانوادگی: #3 \hfill شماره دانشجویی: #4
\vspace{-4mm}
\rule{\textwidth}{1pt}
%\ \\
}

% example environment
\newenvironment{example}
{\smallskip \noindent \emph{مثال:}}
{\hfill $\boxtimes$ \smallskip}

\def\Max{\text{بیشینه کن}}
\def\Min{\text{کمینه کن}}
\def\st{\text{\rl{که}}}

\newtheorem{theorem}{قضیه}
\newtheorem{proposition}{گزاره}
\newtheorem*{claim}{ادعا}
\newtheorem*{lemma}{لم}
\newtheorem{numlemma}{لم}
\newtheorem{corollary}{نتیجه}
\newtheorem*{definition}{تعریف} % Use this for non-trivial definitions.
 %%%%%%%%%%%%%%%%%%%%%%%%%%%%%%%%%%%%%%%%%%%%%%%%%%%%%%%%%%%%%%%%%%%%%%%%%%%%




\begin{document}
%\exercise{8}{لیست‌های مرتب‌ شده}{آئیریا محمدی}
%\exercise{6}{نزدیک‌ترین عنصر}{آئیریا محمدی}
%\exercise{5}{مرتب‌سازی خطی}{آئیریا محمدی}
\exercise{8}{لیست‌های مرتب شده}{آئیریا محمدی}
%الف)
الگریتم زیر را ارائه می‌دهیم

\begin{latin}
\begin{codebox}
	\Procname{$\proc{find k'th biggest(M: MaxHeap, k: int)}$}
		\li let m : MaxHeap
		\li m.add(M.rootNode)
		\li let count = 0
		\li let node : Node
		\li \While $\text{count} < k$ \Then
			\li node = m.pop() \Comment extract max
			\li count = count + 1
			\li m.add(node.leftchild)
			\li m.add(node.rightchild)
		\End
		\li \Return node.value
	\End
\end{codebox}
\end{latin}

\proof{}
بدیهی است.

ب)


%الف)
الگریتم زیر را ارائه می‌دهیم

\begin{latin}
\begin{codebox}
	\Procname{$\proc{find k'th biggest(M: MaxHeap, k: int)}$}
		\li let m : MaxHeap
		\li m.add(M.rootNode)
		\li let count = 0
		\li let node : Node
		\li \While $\text{count} < k$ \Then
			\li node = m.pop() \Comment extract max
			\li count = count + 1
			\li m.add(node.leftchild)
			\li m.add(node.rightchild)
		\End
		\li \Return node.value
	\End
\end{codebox}
\end{latin}

\proof{}
بدیهی است.

ب)


%\section{جایگشت با کمترین عملیات}

\newcommand{\co}[2]{\textcolor{#1}{#2}}

\begin{equation*}
	A = \{1,3,2,6,5,7,4\}
\end{equation*}

مراحل اجرای الگوریتم:
\begin{align*}
	&A = \{\co{red}{1},3,2,6,5,7,\co{blue}{4}\} \\
	&A = \{1,\co{red}{3},2,6,5,7,\co{blue}{4}\} \\
	&A = \{1,3,\co{red}{2},6,5,7,\co{blue}{4}\} \\
	&A = \{1,3,\co{red}{2},\co{green}{6},5,7,\co{blue}{4}\} \\
	&A = \{1,3,\co{red}{2},6,\co{green}{5},7,\co{blue}{4}\} \\
	&A = \{1,3,\co{red}{2},6,5,\co{green}{7},\co{blue}{4}\} \\
	&A = \{1,3,\co{red}{2},6,5,\co{green}{7},\co{blue}{4}\} \\
	&A = \{1,3,2,\co{blue}{4},5,7,6\} \\
	&A = \{\co{red}{1},3,\co{blue}{2},4,5,7,6\} \\
	&A = \{\co{red}{1},\co{green}{3},\co{blue}{2},4,5,7,6\} \\
	&A = \{1,\co{blue}{2},3,4,5,7,6\} \\
	&A = \{1,2,3,4,\co{red}{5},7,\co{blue}{6}\} \\
	&A = \{1,2,3,4,\co{red}{5},\co{green}{7},\co{blue}{6}\} \\
	&A = \{1,2,3,4,5,\co{blue}{6},7\} \\
\end{align*}

\section{
یافتن جایگشت با کمترین عملیات برای هر n دلخواه
}

\begin{latin}
\begin{codebox} \Procname{$\proc{scramble}(A=\{1,\cdots,n\}, start = 1, end = length[A])$} 
\li		x = (end-start+1)
\li		\If x == 1 \Then
\li			\Return
 		\End
\li		i = x/2 - 1
\li		\If x \% 2 = 1 \Then
\li		scramble(start, start+i)
\li		scramble(end-i, end)
\li		swap(A[start+i+1], A[end])	
\li		\Else
\li		scramble(start, start+i-1)
\li		scramble(end-i, end)	
\li		swap(A[start+i], A[end])	
		\End

\end{codebox}
\end{latin}
\par
الف)
هر یک از بخش‌های لیست اندازه 
$k$
دارند و در بدترین حالت در 
$O(k^2)$
مرتب می‌شوند.
جمعا 
$n/k$
لیست داریم پس روی هم 
$O(\frac{n}{k}\cot k^2) = O(nk)$
زمان می‌گیرند.

\par
%ب)
%دو لیست مرتب شده با سایز
%$k$
%با حداکثر 
%$2k-1$
%مقایسه ادغام می‌شوند (تمرین سری ۲).
%
%اگر $m$ لیست مرتب شده با اندازه $k$ داشته باشیم٫ ادغام آن ها
%$T(n = mk) = O(mk\cdot log(mk))$
%به طول می‌انجامد.
%پس اگر 
%$m = \frac{n}{k}$
%داریم:
%\begin{equation*}
%	T(n) = O(\frac{n}{k}\cdot k log(\frac{n}{k}\cdot k)) = O(
%\end{equation*}
ب)
تقسیم یک لیست به $n/k$ لیست مرتب مانند این است که بخشی از عمق الگوریتم مرج سورت را طی کرده باشیم (تا وقتی که اندازه هر لیست k شده باشد). در هر مرحله اندازه زیرلیست‌ها دوبرابر می‌شود پس انگار $log.k$ مرحله جلوییم.
و در نتیجه تعداد عملیات باقی مانده متناسب با عمق باقی مانده است که می‌شود 
$log \cdot n - log \cdot k = log \frac{n}{k}$
.
در هر مرحله نیز از 
$\Theta(n)$ 
کار انجام می‌دهیم پس مرتبه زمانی این ادغام از 
$\Theta(nlog\frac{n}{k})$
خواهد بود.

ج)
k 
نمی‌تواند از 
$\Omega(n)$ 
باشد چرا که 
الگوریتم ما از مرتبه 
$\Omega(n^2)$
خواهد شد.
به ازای 
$k= O(log(n))$
خواهیم داشت
$$T = O(n\cdot logn + n log \frac{n}{logn}) = O(nlogn - n log(log(n))) = O(nlogn)$$

پس $k(n) \in O(logn)$
.

د)
اگر
$a,b$
ثابت باشند
داریم 
$T(n,k) = ank + bnlog\frac{n}{k} = ank + bnlogn - bnlogk$.
قرار می‌دهیم 
$\frac{d}{dk} T = 0$
که می‌دهد 
$$ an + 0 - bn/k = 0 \rightarrow k = b/a $$

یعنی یک مقدار بهینه ثابت برای k وجود داره.
 می‌توانیم برای $n$ های نسبتا بزرگ $k$ های مختلف 
 از
 $O(logn)$
 را امتحان کنیم و بهترین مورد را انتخاب کنیم.

%
%\newpage
%\section{عنوان بخش}
%شروع بحث جلسه فعلی. 
%
%پاراگراف‌ها با یک خط خالی از هم جدا می‌شوند. لازم نیست بین هر دو پاراگراف از \verb|\par|  استفاده کنیم.
%
%می‌توانید برای نوشتن فرمول چندخطی می‌توانید از محیط \verb|align| استفاده کنید:
%\begin{align*} 
%\min \quad \| f \|_{\infty}\\ 
%f \in F_{s,t} 
%\end{align*}
%
%سعی کنید قواعد نگارش فارسی را رعایت کنید. از نیم‌فاصله
%به درستی استفاده کنید. علامت نقل قول در فارسی بدین صورت است «». پس از نقطه و ویرگول و دونقطه و پرانتزبسته‌ای که قبل از نقطه نیست و این‌گونه علامت‌ها، یک فاصله بگذارید.
%
%خوب است معادل انگلیسی اصطلاحات را در پاورقی%
%\LTRfootnote{footnote}
%بیاورید.
%
%برای نوشتن انگلیسی در میان متن فارسی، از دستور \verb+\lr{}+ استفاده کنید. 
%مثلا:
%\lr{Some English text here} 
%در میان متن درست می‌آید.
%
%استفاده از تاکید به صورت 
%\textbf{پررنگ}
%کردن یا 
%\textit{ایتالیک}
%کردن مفید است. 
%
%برای شبه کدها از پکیج
%\lr{clrscode3e}
%استفاده کنید. برای آشنایی با این پکیج
%\lr{clrscode.pdf}
%را مطالعه کنید.
%
%
%\section{محیط‌های مختلف}
%\begin{lemma}
%	یک لم.
%\end{lemma}
%\begin{theorem}
%	\label{thm:sample}
%	یک قضیه. 
%\end{theorem}
%\begin{proof}
%	بدیهی.
%\end{proof}
%
%\begin{example}
%	یک مثال. 
%\end{example}
%
%ارجاع به قضیه 
%\ref{thm:sample}.
%
%ارجاع به مراجع
%\cite{lecture1}
%و
%\cite{CLRS}.
%
%%برای قاطی نشدن متن فارسی و انگلیسی از
%% Enter
%%  زیاد استفاده کنید!
%
%\bibliographystyle{alpha}
%
%\begin{thebibliography}{99}
%	\bibitem{lecture1}
%	جزوه جلسه اوّل
%	\bibitem{dr. ghodsi}
%	قدسی، محمّد. \textit{داده ساختارها و مبانی الگوریتم‌ها}. تهران: فاطمی، ۱۳۹۵
%	
%	\begin{latin}	%english reference
%		
%		%book in MLA style: https://owl.purdue.edu/owl/research_and_citation/mla_style/mla_formatting_and_style_guide/mla_works_cited_page_books.html
%		\bibitem{CLRS}
%		Cormen, Thomas H., et al.
%		\textit{Introduction to Algorithms}. 
%		3rd ed., MIT Press, 2009, pp. 18-22. %page numbers
%	\end{latin}	
%\end{thebibliography}

\end{document}
